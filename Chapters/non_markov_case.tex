\section{Non-Markov Case}


%%%%%%%%%%%%%%%%%%%%%%%%%%%%%%%%%%%%%%%%%%%%%%%%%%%%%%%%%%%%%%%%%%%%%%
%%%%%%%%%%%%%%%%%%%%%%%   PROBLEM DESCRIPTION   %%%%%%%%%%%%%%%%%%%%%%
%%%%%%%%%%%%%%%%%%%%%%%%%%%%%%%%%%%%%%%%%%%%%%%%%%%%%%%%%%%%%%%%%%%%%%%
\subsection{Problem Description}
We consider an $n-$dimensional process $X(t)$ satisfying the following SDE:
\begin{equation}
    \begin{cases}
        \d X(t) &= \big[(A(t) X(t) + B(t)\pi(t)\big] \d t + \sum_{i=1}^d \big[C_i(t) X(t) + D_i(t) \pi(t) \big] \d W_i(t)\\
        X(0) &= x_0 \in \R^n
    \end{cases}
    \label{eq:primal_sde1}
\end{equation}
where $W(t)$ is a $d-$dimensional Brownian motion, $\pi(t) \in \R^m$ is the control, $A_i(t), C_i(t) \in \R^{n \times n}$, and $B_i(t), D_i(t) \in \R^{n \times m}$, for $i = \{ 1, \dots, d\}$.\\

For simplicity, we denote
\begin{align*}
    b(t, X(t), \pi(t))&:= A(t) X(t) + B(t) \pi (t) \in \R^{n}\\
    \sigma(t, X(t), \pi(t)) &:= 
    \begin{bmatrix}
        \begin{pmatrix}
            \\
            \\
            C_1(t) X(t) + D_1(t) \pi(t)\\
            \\
            \\
        \end{pmatrix} 
        & \cdots & 
        \begin{pmatrix}
            \\
            \\
            C_d(t) X(t) + D_d(t) \pi(t)\\
            \\
            \\
        \end{pmatrix}
    \end{bmatrix}
    \in \R^{n \times d}
\end{align*}
The system is then written as: 
\begin{equation}
    \begin{cases}
        \d X(t) &= b(t, X(t),\pi(t)) \d t + \sigma(t, x(t), \pi(t)) \d W(t)\\
        X(0) &= x_0 \in \R^n
    \end{cases}
    \label{eq: primal_sde2}
\end{equation}
The cost functional is given by
\begin{equation}
    J(t, X(t), \pi(t)) := \E \bigg[ \int_{t_0}^T f(t, X(t), \pi(t)) \d t + g(X(T))\bigg],
    \label{eq: cost_func}
\end{equation}
where $f$ is a quadratic function
\begin{align}
    f(t, X(t), \pi(t)) &= \frac12 X^T(t) Q(t) X(t) + X^T(t) S^T(t) \pi(t) + \frac12 \pi^T(t) R(t) \pi(t) \label{eq: f}\\
    g(X(T)) &= \frac12 X^T(T) G(T) X(T) + X^T(T) L(T). \label{eq: g}
\end{align}
The objective is to minimise the cost function over the control:
\begin{equation}
    \text{Minimise } J(t, X(t), \pi(t)) \text{ over } \pi(t) \in \R^m. \label{eq: minimisation_problem}
\end{equation}

%%%%%%%%%%%%%%%%%%%%%%%%%%%%%%%%%%%%%%%%%%%%%%%%%%%%%%%%%%%%%%%%%%%%%%
%%%%%%%%%%%%%%%%%%%%%%%   PRIMAL HJB EQUATION   %%%%%%%%%%%%%%%%%%%%%%
%%%%%%%%%%%%%%%%%%%%%%%%%%%%%%%%%%%%%%%%%%%%%%%%%%%%%%%%%%%%%%%%%%%%%%%

\subsection{Primal Hamilton-Jacobi-Bellman Equation}

\subsubsection{Derivation of Hamilton-Jacobi-Bellman equation}
We transform the minimisation problem to maximisation by noting that
\begin{equation*}
     \inf_{\pi(t) \in \R^m} \E \bigg[ \int_{t_0}^T f(t, X(t), \pi(t)) \d t + g(X(T))\bigg] = - \sup_{\pi(t) \in \R^m} \E \bigg[ \int_{t_0}^T - f(t, X(t), \pi(t)) \d t - g(X(T))\bigg],
\end{equation*}
and denote the value function 
\begin{equation}
    v(t, X(t)) = \sup_{\pi(t) \in \R^m} \E \bigg[ \int_{t_0}^T - f(t, X(t), \pi(t)) \d t - g(X(T))\bigg]
\end{equation}
Consider the time interval $(t, t + h)$ and a constant control $\pi(t) = \pi$. According to the Dynamic programming principle,
\begin{equation}
    v(t, X(t)) \ge \E \bigg[ \int_{t}^{t+h} - f(s, X(s), \pi) \d s + v(t+h, X(t + h)) \bigg],
    \label{eq: 1}
\end{equation}
where we denote $X(s)$ to be the solution of \eqref{eq: primal_sde2} given that we know the value of $X$ at time $t$.\\

Applying Ito's formula between $t$ and $t+h$ we get
\begin{align*}
    v(t+h, X(t+h)) = v(t, X(t)) &+ \int_t^{t+h} \bigg( \frac{\partial v(s, X(s))}{\partial t} + \mathcal{L}^\pi [v(s, X(s)] \bigg) \d s\\
    &+ \underbrace{\int_{t}^{t+h} \big[D_x v(s, X(s))\big]^T \sigma(s, X(s), \pi) \d W(s)}_{\text{(local) martingale}},
\end{align*}
where $\mathcal{L}^\pi[v(t,x)]$ is the generator given by
\begin{equation}
    \mathcal{L}^\pi [v(t, X(t))] 
    = b^T(t, X(t), \pi) D_x v(t, X(t)) + \frac12 \tr\big[\sigma(t, X(t), \pi) \sigma^T(t, X(t), \pi) D_x^2 v(t, X(t))\big]
    \label{eq: generator}
\end{equation}
Substituting into equation \eqref{eq: 1}, we get 
\begin{equation*}
    0 \ge \E \bigg[ \int_{t}^{t + h} \frac{\partial v}{\partial t}(s, X(s)) + \mathcal{L}^\pi[v(s, X(s))] - f(s, X(s), \pi) \d s \bigg]
\end{equation*}
Dividing by $h$ and sending $h$ to $0$, this yields by the mean value theorem 
\begin{equation*}
    0 \ge \frac{\partial v}{\partial t}(t, x) + \mathcal{L}^\pi[v(t,x)] - f(t, x, \pi).
\end{equation*}
Since this is true for any admissible $\pi$, we obtain the inequality
\begin{equation}
    \frac{\partial v}{\partial t} (t, x) + \sup_{\pi \in \R^m} [\mathcal{L}^\pi v(t,x) - f(t, x, \pi)] \le 0.
    \label{eq: 2}
\end{equation}
On the other hand, suppose that $\pi^*$ is an optimal control. Then by the dynamic programming principle, 
\begin{equation}
    v(t,x) = \E\bigg[ \int_{t}^{t+h} - f(s, X^*(s), \pi^*(s)) \d s + v(t+h, X^*(t+h)) \bigg],
\end{equation}
where $X^*$ is the solution to the initial SDE \eqref{eq: primal_sde2} with control $\pi^*$ starting from $x$ at time $t$. By similar reasoning, we get 
\begin{equation*}
    \frac{\partial v}{\partial t} (t,x) + \mathcal{L}^{\pi^*} [v(t,x)] - f(t, x, \pi^*) = 0
\end{equation*}
which combined with \eqref{eq: 2} suggests that $v$ should satisfy
\begin{equation}
    \frac{\partial v}{\partial t} (t,x) + \sup_{\pi \in \R^m} [\mathcal{L}^{\pi(t)} [v(t,x)] - f(t, x, \pi)] = 0, \quad \forall (t,x) \in [t_0, T) \times \R^n.
    \label{eq: hjb_inf}
\end{equation}
with the terminal condition:
\begin{equation*}
    v(T,x) = - g(x) = - \frac12 x^T G(T) x - x^T L(T), \quad \forall x \in \R^n.
\end{equation*}
Equation \eqref{eq: hjb_inf} is called the Hamilton-Jacobi-Bellman equation. The supremum can be found by setting the derivative with respect to $\pi$ to zero. The derivative of the generator $\mathcal{L}^{\pi}$, $D_\pi [\mathcal{L}^\pi]  \in \R^m$, is given by:
\begin{equation}
    D_{\pi} \big[ \mathcal{L}^{\pi}[v(t, x)]\big] = D_{\pi} \big[ b(t,x, \pi)^T D_x[v(t,x)] \big] + D_\pi \bigg[ \frac12 \tr (\sigma(t, x,\pi) \sigma^T(t, x, \pi) D^2_x[v(t,x)])\bigg].
    \label{eq: generator_derivative}
\end{equation}
We have that 
\begin{align*}
    D_\pi \big[ b^T(t,x, \pi) D_x[v(t,x)] \big] 
    &= D_\pi \big[(x^T A^T(t) + \pi^T(t) B^T(t)) D_x[v(t,x)] \big]\\
    &= B^T(t) D_x [v(t,x)] \numberthis \label{eq: derivative_1}
\end{align*}
The latter derivative in \eqref{eq: generator_derivative} is given by:
\begin{align*}
    D_\pi \bigg[ \frac12 \tr[\sigma(t,x,\pi) \sigma^T(t,x,\pi) D_x^2[v]] \bigg]
    &= \frac12 D_\pi \bigg[ \tr\bigg[ \sum_{i=1}^d (C_i x + D_i \pi)(C_i x + D_i \pi)^T D_x^2[v] \bigg] \bigg]\\
    &= \frac12 \sum_{i=1}^d D_\pi \big[ \tr[(C_i x + D_i \pi)(C_i x + D_i \pi)^T D_x^2[v]] \big]\\
    &= \frac12 \sum_{i=1}^d D_\pi \big[(C_i x + D_i \pi)^T D_x^2[v](C_i x + D_i \pi)\big] \\
    &= \sum_{i=1}^d D_i^T D_x^2[v(t,x)] (C_i x + D_i \pi) \numberthis \label{eq: derivative_2}
\end{align*}
The derivative of $f(t,x,\pi)$ with respect to $\pi$ is 
\begin{equation}
    D_\pi f(t, x, \pi) = S x + R \pi \label{eq: derivative_3}
\end{equation}
Combining the three equations,\eqref{eq: derivative_1}, \eqref{eq: derivative_2}, \eqref{eq: derivative_3}, we get that
\begin{equation*}
    D_\pi [\mathcal{L}^\pi(t)[v(t,x)] - f(t,x,\pi)]
    = B^T D_x[v(t,x)] + \sum_{i=1}^d D_i^T D_x^2[v(t,x)] (C_i x + D_i \pi) - S x - R \pi
\end{equation*}
Setting this to zero, we get
\begin{equation}
    \pi^\ast = \bigg(\sum_{i=1}^d D_i^T D_x^2[v(t,x)] D_i - R\bigg)^{-1} \bigg(S x - B^T D_x[v(t,x)] - \sum_{i=1}^d D_i^T D_x^2[v(t,x)] C_i x\bigg) \label{eq: control_optimal_primal_hjb}
\end{equation}

We now substitute \eqref{eq: control_optimal_primal_hjb} into \eqref{eq: hjb_inf} to get:
\begin{equation*}
    \frac{\partial v}{\partial t} + b(t, x, \pi^\ast)^T D_x[v] + \frac12 \tr \big[ \sigma(t, x, \pi^\ast) \sigma^T(t, x, \pi^\ast) D_x^2[v]\big] - \frac12 x^T Q x - \frac12 x^T S^T \pi^\ast - \frac12 {\pi^\ast}^T S x - \frac12 {\pi^\ast}^T R \pi^\ast = 0
\end{equation*}
As $D_x^2[v]$ is a symmetric matrix, we can write
\begin{align*}
    \tr \big[\sigma(t, x, \pi^\ast) \sigma^T(t, x, \pi^\ast) D_x^2[v] \big] &= \sum_{i=1}^d \tr \big[ (C_i x + D_i \pi^\ast)(C_i x + D_i \pi^\ast)^T D_x^2[v]  \big]\\
    &= \sum_{i=1}^d (C_i x + D_i \pi^\ast)^T D_x^2[v](C_i x + D_i \pi^\ast),
\end{align*}
we get the Hamilton-Jacobi-Bellman equation 
\begin{align*}
    \frac{\partial v}{\partial t} + (A x + B \pi^\ast)^T D_x[v] &+ \frac12 \sum_{i=1}^d (C_i x + D_i \pi^\ast)^T D_x^2[v](C_i x + D_i \pi^\ast)\\
    &- \frac12 x^T Q x - \frac12 x^T S^T \pi^\ast - \frac12 {\pi^\ast}^T S x - \frac12 {\pi^\ast}^T R \pi^\ast = 0 \numberthis \label{eq: hjb_primal}
\end{align*}
where $\pi^\ast$ is as in \eqref{eq: control_optimal_primal_hjb} and the terminal condition is given by
\begin{equation*}
    v(T, x) = - g(x) = - \frac12 x^T G(T) x - x^T L(T). 
\end{equation*}

\subsubsection{Solving the Primal Hamilton-Jacobi-Bellman Equation}
We assume that $v(t,x)$ is a quadratic function in $x$ and we use the ansatz
\begin{equation}
    v(t,x) = \frac12 x^T P(t) x + x^T M(t) + N(t), 
    \label{eq: ansatz_primal_hjb}
\end{equation}
with terminal conditions:
\begin{equation}
    P(T) = -G(T), \quad M(T) = - L(T), \quad N(T) = 0. \label{eq: terminal_conditions_pirmal_hjb}
\end{equation}
Then 
\begin{align*}
    &\frac{\partial v}{\partial t}(t,x) = \frac12 x^T \dot{P}(t) x + x^T \dot{M}(t) + \dot{N}(t)\\
    &D_x[v(t,x)] = P(t) x + M(t)\\
    &D^2_x[v(t,x)] = P(t).
\end{align*}
Substituting in \eqref{eq: control_optimal_primal_hjb} we get the optimal control
\begin{equation*}
    \pi^\ast = \bigg(\sum_{i=1}^d D_i^T P D_i - R\bigg)^{-1} \bigg(S x - B^T P x - B^T M - \sum_{i=1}^d D_i^T P C_i x\bigg)
\end{equation*}
We can write this as 
\begin{equation*}
    \pi^\ast = \vartheta_1 x + \kappa_1,
\end{equation*}
where
\begin{equation}
    \vartheta_1 = \bigg(\sum_{i=1}^d D_i^T P D_i - R\bigg)^{-1} \bigg(S - B^T P - \sum_{i=1}^d D_i^T P C_i \bigg), \quad \kappa_1 = -\bigg(\sum_{i=1}^d D_i^T P D_i + R\bigg)^{-1} B^T M \label{eq: theta_kappa_primal_hjb}
\end{equation}
Substituting this into \eqref{eq: hjb_primal}, we get 
\begin{align*}
    &\frac{\partial v}{\partial t} + (A x + B (\vartheta_1 x + \kappa_1))^T D_x[v] + \frac12 \sum_{i=1}^d (C_i x + D_i (\vartheta_1 x + \kappa_1))^T D_x^2[v](C_i x + D_i (\vartheta_1 x + \kappa_1))\\
    &- \frac12 x^T Q x - \frac12 x^T S^T (\vartheta_1 x + \kappa_1) - \frac12 {(\vartheta_1 x + \kappa_1)}^T S x - \frac12 {(\vartheta_1 x + \kappa_1)}^T R (\vartheta_1 x + \kappa_1) = 0 \implies \\
    &\frac12 x^T \dot{P} x + x^T \dot{M} + \dot{N} + (x^T A^T + x^T \vartheta_1^T B^T + \kappa_1^T B^T)(P x + M)\\
    &+ \frac12 \sum_{i=1}^d (x^T C_i^T + x^T \vartheta_1^T D_i^T + \kappa_1^T D_i^T)P ( C_i x + D_i \vartheta_1 x + D_i \kappa_1)\\
    &- \frac12 x^T Q x - \frac12 x^T S^T (\vartheta_1 x + \kappa_1) - \frac12 {(x^T \vartheta_1^T  + \kappa_1^T)} S x - \frac12 {(x^T \vartheta_1^T + \kappa_1^T)} R (\vartheta_1 x + \kappa_1) = 0
\end{align*}
Rewriting this, we get
\begin{align*}
    &x^T\bigg[ \frac12 \dot{P} + \frac12 A^T P + \frac12 P A + \frac12 \vartheta_1^T B^T P + \frac12 P B \vartheta_1 + \frac12 \sum_{i=1}^d (C_i^T + \vartheta_1^T D_i^T)P(C_i + D_i \vartheta_1)\\
    &- \frac12 Q - \frac12 \vartheta_1^T S - \frac12 S^T \vartheta_1 - \frac12 \vartheta_1^T R \vartheta_1 \bigg]x + x^T \bigg[ \dot{M} + A^T M + P B \kappa_1 + \vartheta_1^T B^T M + \\
    &\sum_{i=1}^d (C_i^T + \vartheta_1^T D_i^T)P D_i \kappa_1 -  S^T \kappa_1 - \vartheta_1^T R \kappa_1 \bigg] + \dot{N} + \kappa_1^T B^T M + \frac12 \sum_{i=1}^d \kappa_1^T D_i^T P D_i \kappa_1 - \frac12 \kappa_1^T R \kappa_1 = 0
\end{align*}
This equation must equal zero for all $x$, hence the coefficients in front of the quadratic term, as well as $x$ and the free coefficient must be zero. Setting the coefficients to zero, we get the system
\begin{align*}
     \frac12 \dot{P} + \frac12 A^T P + \frac12 P A + \frac12 \vartheta_1^T B^T P + \frac12 P B \vartheta_1 + &\frac12 \sum_{i=1}^d (C_i^T + \vartheta_1^T D_i^T)P(C_i + D_i \vartheta_1)\\
     -& \frac12 Q - \frac12 \vartheta_1^T S - \frac12 S^T \vartheta_1 - \frac12 \vartheta_1^T R \vartheta_1 = 0 \numberthis \label{eq: primal_hjb_ricatti_1}\\
     \dot{M} + A^T M + P B \kappa_1 + \vartheta_1^T B^T M + \sum_{i=1}^d (C_i^T + \vartheta_1^T D_i^T&)P D_i \kappa_1 -  S^T \kappa_1 - \vartheta_1^T R \kappa_1 = 0 \numberthis \label{eq: primal_hjb_ricatti_2}\\ 
     \dot{N} + \kappa_1^T B^T M + \frac12 \sum_{i=1}^d \kappa_1^T D_i^T P D_i \kappa_1 - \frac12 \kappa_1^T R \kappa_1& = 0, \numberthis
\end{align*}
where $\vartheta_1$ and $\kappa_1$ are as in \eqref{eq: theta_kappa_primal_hjb} and the terminal conditions are as in \eqref{eq: terminal_conditions_pirmal_hjb}.



%%%%%%%%%%%%%%%%%%%%%%%%%%%%%%%%%%%%%%%%%%%%%%%%%%%%%%%%%%%%%%
%%%%%%%%%%%%%%%%%%%%%%%   PRIMAL BSDE   %%%%%%%%%%%%%%%%%%%%%%
%%%%%%%%%%%%%%%%%%%%%%%%%%%%%%%%%%%%%%%%%%%%%%%%%%%%%%%%%%%%%%


\subsection{Primal BSDE}



We define the Hamiltonian $\mathcal{H} :\Omega \times [t_0, T] \times \R^n \times K \times R^n \times \R^{n\times d} \to \R$ by
\begin{align*}
    \mathcal{H}(t, x, \pi, p, q) 
    &= b^T p + \tr (\sigma^T q) - f(t,x,\pi)\\
    &= x^T A^T p + \pi^T B^T p + \sum_{i=1}^d \bigg( x^T C_i^T q_i +  \pi^T D_i^T q_i \bigg) - \frac12 x^T Q x - \frac12 x^T S^T \pi - \frac12 \pi^T S x- \frac12 \pi^T R \pi, \numberthis \label{eq: hamiltonian_primal}
\end{align*}
where we denote by $q_i \in \R^n$ the $i^{\text{th}}$ column of the matrix $q \in \R^{n \times d}$. We assume that $\mathcal{H}$ is differentiable in $x$ with derivative $D_x \mathcal{H} \in \R^{n}$.\\

The theorem below states the Stochastic Maximum Principle for the minimisation problem \eqref{eq: minimisation_problem}.
\begin{theorem*}
    Let $\hat{\pi} \in \R^m$ be an admissible control. Then $\hat{\pi}$ is optimal control from problem \eqref{eq: minimisation_problem} if and only if the solution $(\hat{X}, \hat{p}, \hat{q})$ of the FBSDE
    \begin{equation}
        \begin{cases}
            \d \hat{X} &= (A X + B \pi) \d t + \sum_{i=1}^d (C_i \hat{X} + D_i \hat{\pi}) \d W_i\\
            \hat{X}(t_0) &= x_0\\
            \d \hat{p} &= -D_x[\mathcal{H}(t, X(t), \pi(t), p(t), q(t))] \d t + \sum_{i=1}^d q_i(t) \d W_i(t)\\
            %&= -\big[ A^T p + \sum_{i=1}^d C_i^T q_i - Q X(t) - S^T \pi \big] \d t + \sum_{i=1}^d q_i(t) \d W_i(t)\\
            \hat{p}(T) &= - D_x g(\hat{X}(T))= - G(T)\hat{X}(T) - L(T)
        \end{cases}
        \label{eq: fbsde_primal}
    \end{equation}
    satisfies the condition
    \begin{equation*}
    \mathcal{H} (t, \hat{X}(t),\hat{\pi}(t), \hat{Y}(t), \hat{Z}(t)) = \max_{\pi \in K} \mathcal{H} (t, \hat{X}(t), \pi(t), \hat{Y}(t), \hat{Z}(t)), \quad t_0 \le t \le T.
\end{equation*}
\end{theorem*}
Since we have no constraints on the control $\pi$, i.e. $K = \R^m$, we find the optimal $\pi$ by setting
\begin{equation*}
    D_\pi \mathcal{H}(t, X(t), {\pi}(t), p(t), q(t)) = 0.
\end{equation*}
We have
\begin{equation*}
    D_\pi \mathcal{H}(t, X(t), \pi(t), p(t), q(t)) = B^T p + \sum_{i=1}^d D_i^T  q_i - S X - R\pi
\end{equation*}
so
\begin{equation}
    B^T p + \sum_{i=1}^d D_i^T q_i - S X -  R\pi = 0
    \label{eq: hamiltonian_condition_primal}
\end{equation}
We try an ansatz for the control $\pi$ of the form 
\begin{equation}
    \pi = \vartheta_2 X + \kappa_2,
\end{equation}
and an ansatz for $p$ of the form:
\begin{equation*}
    p = \varphi(t) X(t) + \psi(t),
\end{equation*}
where $\varphi(t) \in \R^{n \times n}$ and $\psi(t) \in R^n$. Substituting the control in the Hamiltonian \eqref{eq: hamiltonian_primal} we get
\begin{align*}
    \mathcal{H} = X^T A^T p + (\vartheta_2 X + \kappa_2)^T B^T p + \sum_{i=1}^d \bigg( X^T C_i^T q_i +  (\vartheta_2 X + \kappa_2)^T D_i^T q_i \bigg)
    - \frac12 X^T Q X\\ - \frac12 X^T S^T (\vartheta_2 X + \kappa_2) - \frac12 (\vartheta_2 X + \kappa_2)^T S X
    - \frac12 (\vartheta_2 X + \kappa_2)^T R (\vartheta_2 X + \kappa_2) \numberthis \label{eq: hamiltonian_primal_no_control}
\end{align*}
The derivative of the Hamiltonian is then 
\begin{equation}
    D_x[\mathcal{H}] = A^T p + \vartheta_2^T B^T p + \sum_{i=1}^d C_i^T q_i + \sum_{i=1}^d \vartheta_2^T D_i^T q_i - QX - 2S^T\vartheta_2 X - S^T \kappa_2 - \vartheta_2^T R \vartheta_2 X - \vartheta_2^T R \kappa_2 \label{eq: hamiltonian_derivative_primal}
\end{equation}
%Then the BSDE \eqref{eq: fbsde_primal} becomes 
%\begin{align}
%    \d p = \bigg[ -A^T \varphi X - A^T\psi - \sum_{i=1}^d C_i^T q_i + Q X + S^T \vartheta_2 X + S^T \kappa_2 \bigg] \d t + \sum_{i=1}^d q_i &\d W_i
%    \label{eq: BSDE_primal_no_control}
%\end{align}
On the other hand, if we let $p = f(t,x) = \varphi(t) x + \psi(t)$, then we can apply Ito's formula:
\begin{align*}
    \d p &= \frac{\partial f}{\partial t} \d t + (D_x[f])^T\d X + \frac{1}{2} (\d X)^T D_x^2[f] \d X\\
    &= (\dot{\varphi} X + \dot{\psi}) \d t + \varphi \d X\\
    &= (\dot{\varphi} X + \dot{\psi}  + \varphi b(t,X, \pi) ) \d t + \varphi \sigma(t, X, \pi) \d W\\
    &= \big[\dot{\varphi} X + \dot{\psi} + \varphi A X + \varphi B \pi \big] \d t + \varphi \sum_{i=1}^d (C_i X + D_i \pi) \d W_i\\
    &= \bigg[\dot{\varphi} X + \dot{\psi} + \varphi A X + \varphi B \vartheta_2 X + \varphi B \kappa_2\bigg] \d t + \sum_{i=1}^d \varphi (C_i X + D_i \vartheta_2 X + D_i \kappa_2) \d W_i \numberthis \label{eq: bsde_primal_ito}
\end{align*}
Equating the coefficients of \eqref{eq: bsde_primal_ito} with \eqref{eq: fbsde_primal}, we get the system
\begin{align}
    &\dot{\varphi} X + \dot{\psi} + \varphi A X + \varphi B \vartheta_2 X + \varphi B \kappa_2 = - D_x[\mathcal{H}(t, X(t), \pi(t), p(t), q(t))] \label{eq: system_1}\\
    %-A^T \varphi X - A^T\psi - \sum_{i=1}^d C_i^T q_i + Q X + S^T \vartheta_2 X + S^T \kappa_2\\
    &\varphi (C_i X + D_i \vartheta_2 X + D_i \kappa_2) = q_i \quad i \in \{1, \dots, d \}\\
    &B^T \varphi X + B^T \psi + \sum_{i=1}^d D_i^T q_i - S X - R(\vartheta_1 X + \kappa_1) = 0, \label{eq: system_2}
\end{align}
where the third equation is the Hamiltonian condition \eqref{eq: hamiltonian_condition_primal}. We now substitute $q_i$ from the second equation into \eqref{eq: hamiltonian_derivative_primal} and \eqref{eq: system_2}, so that our system becomes
\begin{align*}
    \dot{\varphi}(t) X(t) + \dot{\psi}(t) + \varphi(t) A X(t) + \varphi B \vartheta_2 X + \varphi B \kappa_2
    = -A^T \varphi X - A^T \psi - \vartheta_2^T B^T \varphi X - \vartheta_2^T B^T \psi\\ 
    - \sum_{i=1}^d C_i^T \varphi (C_i X + D_i \vartheta_2 X + D_i \kappa_2) - \sum_{i=1}^d \vartheta_2^T D_i^T \varphi (C_i X + D_i \vartheta_2 X + D_i \kappa_2)\\
    + QX + 2S^T\vartheta_2 X + S^T \kappa_2 + \vartheta_2^T R \vartheta_2 X + \vartheta_2^T R \kappa_2 \numberthis \label{eq: equal_coeff_primal}\\
    B^T \varphi X + B^T \psi + \sum_{i=1}^d D_i^T \big[\varphi (C_i X + D_i \vartheta_2 X + D_i \kappa_2) \big] - S X - R\vartheta_2 x - R \kappa_2 = 0 \numberthis \label{eq: system_3}
\end{align*}
From \eqref{eq: system_3}, we get 
\begin{equation}
    \pi^\ast = \vartheta_2 X + \kappa_2 = \bigg[ \sum_{i=1}^d D_i^T \varphi D_i -  R \bigg]^{-1} \bigg( S X - B^T \varphi X - B^T \psi - \sum_{i=1}^d D_i^T \varphi C_i X \bigg),
\end{equation}
i.e., 
\begin{equation}
    \vartheta_2 = \bigg[ \sum_{i=1}^d D_i^T \varphi D_i - 2 R \bigg]^{-1} \bigg( S - B^T \varphi - \sum_{i=1}^d D_i^T \varphi C_i \bigg), \quad \kappa_2 = -  \bigg[ \sum_{i=1}^d D_i^T \varphi D_i - 2 R \bigg]^{-1} B^T \psi.\label{eq: control_parameters_primal}
\end{equation}
We rewrite equation \eqref{eq: equal_coeff_primal} as
\begin{align*}
    \bigg[\dot{\varphi} + \varphi A + A^T \varphi + \varphi B \vartheta_2 + \vartheta_2^T B^T \varphi + \sum_{i=1}^d (C_i^T + \vartheta_2^T D_i^T) \varphi (C_i + D_i \vartheta_2) - Q - S^T \vartheta_2 - \vartheta_2^T S - \vartheta_2^T R \vartheta_2\bigg]X  \\
    + \big[ \dot{\psi} + \varphi B \kappa_2 + A^T \psi + \vartheta_2^T B^T \psi + \sum_{i=1}^d C_i^T \varphi D_i \kappa_2 + \sum_{i=1}^d \vartheta_2^T D_i^T \varphi D_i \kappa_2 - S^T \kappa_2 - \vartheta_2^T R \kappa_2 \big]= 0
\end{align*}
Since this must be true for all $X$, the coefficient in front of $X$ must be equal to zero, so we get 
\begin{align}
    &\dot{\varphi} + \varphi A + A^T \varphi + \varphi B \vartheta_2 + \vartheta_2^T B^T \varphi + \sum_{i=1}^d (C_i^T + \vartheta_2^T D_i^T) \varphi (C_i + D_i \vartheta_2) - Q - S^T \vartheta_2 - \vartheta_2^T S - \vartheta_2^T R \vartheta_2  = 0  \label{eq: primal_bsde_solution_1}\\
    &\dot{\psi} + \varphi B \kappa_2 + A^T \psi + \vartheta_2^T B^T \psi + \sum_{i=1}^d C_i^T \varphi D_i \kappa_2 + \sum_{i=1}^d \vartheta_2^T D_i^T \varphi D_i \kappa_2 - S^T \kappa_2 - \vartheta_2^T R \kappa_2 = 0, \label{eq: primal_bsde_solution_2}
\end{align}
where $\vartheta_2$ and $\kappa_2$ are as in \eqref{eq: control_parameters_primal}.


%%%%%%%%%%%%%%%%%%%%%%%%%%%%%%%%%%%%%%%%%%%%%%%%%%%%%%%%%%%%%%%%%%%%%
%%%%%%%%%%%%%%%%%%%%%%   EQUIVALENCE HJB AND BSDE %%%%%%%%%%%%%%%%%%%
%%%%%%%%%%%%%%%%%%%%%%%%%%%%%%%%%%%%%%%%%%%%%%%%%%%%%%%%%%%%%%%%%%%%%


\subsection{Equivalence of Primal HJB and Primal BSDE}
From the Primal HJB, we get
\begin{equation}
    \pi^\ast = \bigg[\sum_{i=1}^d D_i^T P D_i - R\bigg]^{-1} \bigg(S x - B^T P x - B^T M - \sum_{i=1}^d D_i^T P C_i x\bigg)
\end{equation}
and from the Primal BSDE:
\begin{equation}
    \pi^\ast = \bigg[ \sum_{i=1}^d D_i^T \varphi D_i -  R \bigg]^{-1} \bigg( S X - B^T \varphi X - B^T \psi - \sum_{i=1}^d D_i^T \varphi C_i X \bigg),
\end{equation}
Comparing, we get the relation
\begin{equation*}
    \varphi = P, \quad \psi = M. 
\end{equation*}
The ODE from the Primal BSDE for $\varphi$ is \eqref{eq: primal_bsde_solution_1}. Substituting $\varphi = P$ and $\vartheta_2 = \vartheta_1$ we get 
\begin{equation*}
    \dot{P} + P A + A^T P + P B \vartheta_1 + \vartheta_1^T B^T P + \sum_{i=1}^d (C_i^T + \vartheta_1^T D_i^T) P (C_i + D_i \vartheta_1) - Q - S^T \vartheta_1 - \vartheta_1^T S - \vartheta_1^T R \vartheta_1  = 0,
\end{equation*}
which is equal to twice the ODE for $P$ from the primal HJB \eqref{eq: primal_hjb_ricatti_1}, i.e. 
\begin{align*}
    \frac12 \dot{P} + \frac12 A^T P + \frac12 P A + \frac12 \vartheta_1^T B^T P + \frac12 P B \vartheta_1 + &\frac12 \sum_{i=1}^d (C_i^T + \vartheta_1^T D_i^T)P(C_i + D_i \vartheta_1)\\
     -& \frac12 Q - \frac12 \vartheta_1^T S - \frac12 S^T \vartheta_1 - \frac12 \vartheta_1^T R \vartheta_1 = 0.
\end{align*}
Similarly, substituting $\varphi=P, \psi = M$, $\vartheta_2 = \vartheta_1$ and $\kappa_2 = \kappa_1$ into \eqref{eq: primal_bsde_solution_2} we get
\begin{align*}
    \dot{M} + P B \kappa_1 + A^T M + \vartheta_1^T B^T M + \sum_{i=1}^d C_i^T P D_i \kappa_1 + \sum_{i=1}^d \vartheta_1^T D_i^T P D_i \kappa_1 - S^T \kappa_1 - \vartheta_1^T R \kappa_1 = 0
\end{align*}
and we will get exactly \eqref{eq: primal_hjb_ricatti_2}:
\begin{align*}
    \dot{M} + A^T M + P B \kappa_1 + \vartheta_1^T B^T M + \sum_{i=1}^d (C_i^T + \vartheta_1^T D_i^T&)P D_i \kappa_1 -  S^T \kappa_1 - \vartheta_1^T R \kappa_1 = 0 
\end{align*}
So the ODEs for solving $P, M$ and $\varphi, \psi$ are identical, hence we have equivalence between the two methods.yop
%%%%%%%%%%%%%%%%%%%%%%%%%%%%%%%%%%%%%%%%%%%%%%%%%%%%%%%%%%%%%%%%%%%%%%%
%%%%%%%%%%%%%%%%%%%%%%%%%%%   DUAL PROBLEM %%%%%%%%%%%%%%%%%%%%%%%%%%%%
%%%%%%%%%%%%%%%%%%%%%%%%%%%%%%%%%%%%%%%%%%%%%%%%%%%%%%%%%%%%%%%%%%%%%%%


\subsection{Dual Problem}
The dual process $Y$ satisfies the following SDE:
\begin{equation}
    \begin{cases}
        \d Y(t) &= \big[ \alpha(t) - A(t)^T Y(t) - \sum_{i=1}^d C_i(t)^T \beta_i(t)\big]\d t + \sum_{i=1}^d \beta_i(t) \d W_i(t)\\
        Y(t_0) &= y, \label{eq: y_sde}
    \end{cases}
\end{equation}
where $\alpha, \beta_i$ and $y$ are all to be determined. There is a unique solution to the SDE for given $(y, \alpha, \beta_1, \dots, \beta_d)$. \\

Using Ito's lemma to $X(t)^T Y(t)$, we get
\begin{align*}
    \d X^T Y &= \bigg(X^T \alpha + \pi^T B^T Y + \pi^T \sum_{i=1}^d D_i^T \beta_i \bigg) \d t + \sum_{i=1}^d \bigg(X^T \beta_i + Y^T (C_i X + D_i \pi)\bigg) \d W_i,
\end{align*}
The process $X^T(t)Y(t) - \int_{t_0}^t [X^T(s) \alpha(s) + \pi^T(s) (B^T Y + \sum_{i=1}^d D_i^T \beta_i)] \d s$ is a local martingale and a supermartingale if it is bounded below by an integrable process, which gives 
\begin{equation}
    \E \bigg[ X^T(T) Y(T) - \int_{t_0}^T \bigg(X^T \alpha + \pi^T B^T Y + \pi^T\sum_{i=1}^d D_i^T \beta_i) \bigg) \d s \bigg] \le x^T y. \label{eq: dual_1}
\end{equation}
The optimisation problem can be written equivalently as 
\begin{equation}
    \max_\pi \E \bigg[ -\int_{t_0}^T f(t,x, \pi) \d t - g(X(T)) \bigg].
\end{equation}
Define the dual functions $\phi : [t_0, T] \times \R^n \times \R^m \to \R$ by
\begin{equation}
    \phi(t, \alpha, \beta) = \sup_{x, \pi} \big\{x^T \alpha + \pi^T \beta - f(t, x, \pi) \big\} \label{eq: phi_1}
\end{equation}
and $h: \R^n \to \R$ by
\begin{equation}
    h(y) = \sup_x \big\{-x^T y - g(x)\big\}. 
    \label{eq: h_1}
\end{equation}
Substituting $f$ and $g$ from \eqref{eq: f} and \eqref{eq: g}, we can find the supremums by setting the derivatives to zero. We get
\begin{equation*}
    \phi(t, \alpha, \beta) = \sup_{x, \pi} \bigg\{
    \begin{bmatrix}
        x\\
        \pi
    \end{bmatrix}^T
    \begin{bmatrix}
        \alpha\\
        \beta
    \end{bmatrix} - \frac12
    \begin{bmatrix}
        x\\
        \pi
    \end{bmatrix}^T
    \begin{bmatrix}
        Q & S^T\\
        S & R
    \end{bmatrix}
    \begin{bmatrix}
        x\\
        \pi
    \end{bmatrix}
    \bigg\},
\end{equation*}
so setting the derivative to zero, we get
\begin{equation*}
    \begin{bmatrix}
        \alpha\\
        \beta
    \end{bmatrix} - 
    \begin{bmatrix}
        Q & S^T\\
        S & R
    \end{bmatrix}
    \begin{bmatrix}
        x\\
        \pi
    \end{bmatrix}
    = 0 \implies 
    \begin{bmatrix}
        x^\ast\\
        \pi^\ast
    \end{bmatrix} = 
    \begin{bmatrix}
        Q & S^T\\
        S & R
    \end{bmatrix}^{-1}
    \begin{bmatrix}
        \alpha\\
        \beta
    \end{bmatrix}.
\end{equation*}
Therefore
\begin{align*}
    \pi^\ast &= \big[ S Q^{-1}S^T - R \big]^{-1}(S Q^{-1} \alpha - \beta)\\
    x^\ast &= Q^{-1} (\alpha - S^T \pi^\ast)
\end{align*}
Then $\phi$ is given by
\begin{equation*}
    \phi(t, \alpha, \beta) = 
    \frac12
    \begin{bmatrix}
        \alpha\\
        \beta
    \end{bmatrix}^T
    \begin{bmatrix}
        Q & S^T\\
        S & R
    \end{bmatrix}^{-1}
    \begin{bmatrix}
        \alpha\\
        \beta
    \end{bmatrix}. 
\end{equation*}
Denoting
\begin{equation*}\begin{bmatrix}
        \tilde{Q} & \tilde{S}^T\\
        \tilde{S} & \tilde{R}
    \end{bmatrix}
    =
    \begin{bmatrix}
        Q & S^T\\
        S & R
    \end{bmatrix}^{-1},
\end{equation*}
where
\begin{align}
    &\tilde{Q} = Q^{-1} - Q^{-1} S^T (S Q^{-1} S^T - R)^{-1}S Q^{-1} \label{eq: tilde_q}\\
    &\tilde{R} = R^{-1} - R^{-1} S (S^T R^{-1}S - Q^{-1})^{-1}S^T R^{-1} \label{eq: tilde_r}\\
    &\tilde{S} = (S Q^{-1} S^T - R)^{-1}S Q^{-1} = R^{-1}S(S^TR^{-1}S - Q)^{-1} \label{eq: tilde_s}
\end{align}
we get
\begin{equation}
    \phi(t, \alpha, \beta) = \frac12 \alpha^T \tilde{Q} \alpha + \alpha^T \tilde{S}^T \beta + \frac12 \beta^T \tilde{R} \beta \label{eq: phi}
\end{equation}
Similarly, 
\begin{equation*}
    D_x \big[-x^T y - \frac12 x^T G x - x^T L \big] = -y - Gx - L \implies x^\ast = - G^{-1} (y + L).
\end{equation*}
Then $h(y)$ is given by
\begin{align*}
    h(y) &= (y^T + L^T) G^{-1} y - \frac12 (y^T + L^T)G^{-1}(y + L) + (y^T + L^T) G^{-1} L\\
    &= \frac12 \big[ y^T G^{-1} y +  L^T G^{-1} y + y^T G^{-1} L + L^T G^{-1} L\big]\\
    &= \frac12 (y^T + L^T)G^{-1}(y + L)\numberthis \label{eq: h}
\end{align*}
Combining \eqref{eq: dual_1}, \eqref{eq: phi_1}, \eqref{eq: h_1}, we get the following inequality:
\begin{equation*}
    \max_\pi \E \bigg[ -\int_{t_0}^T f(t,x,\pi) \d t - g(X(T)) \bigg] \le \min_{y, \alpha, \beta_1, \dots, \beta_d} \bigg[ x^T y + \E\bigg[\int_{t_0}^T \phi\bigg(t,\alpha, B^T Y + \sum_{i=1}^d D_i^T \beta_i\bigg) \d t + h(Y(T)) \bigg] \bigg].
\end{equation*}
The dual control problem is defined by
\begin{equation}
    \min_{y, \alpha, \beta_1, \dots, \beta_d} \bigg[ x^T y + \E\bigg[\int_{t_0}^T \phi\bigg(t,\alpha, B^T Y + \sum_{i=1}^d D_i^T \beta_i\bigg) \d t + h(Y(T)) \bigg] \bigg], \label{eq: dual_control_problem}
\end{equation}
where $Y$ satisfies \eqref{eq: y_sde}. Problem \eqref{eq: dual_control_problem} can be solved in two steps: first, for fixed $y$, solve a stochastic control problem
\begin{equation*}
    V(y) = \min_{ \alpha, \beta_1, \dots, \beta_d} \E\bigg[\int_{t_0}^T \phi \bigg(t, \alpha, B^T Y + \sum_{i=1}^d D_i^T \beta_i \bigg) \d t + h(Y(T)) \bigg],
\end{equation*}
and second, solve a static optimisation problem
\begin{equation}
    \min_y (x^T y + V(y)).
\end{equation}
\subsection{Dual Hamilton-Jacobi-Bellman Equation}
\subsubsection{Deriving the Dual HJB}
Denote the dual value function by
\begin{equation}
    \tilde{v}(t, Y(t)) = \sup_{\alpha, \beta_1, \dots, \beta_d} \E \bigg[ - \int_{t_0}^T \phi\big(t, \alpha, B^T Y + \sum_{i=1}^d D_i^T \beta_i \big) \d t - h(Y(T)) \bigg]
\end{equation}
Then similar to the primal problem, the dual HJB equation is given by
\begin{equation}
    \frac{\partial \tilde{v}}{\partial t} (t, y) + \sup_{\alpha, \beta_1, \dots, \beta_d} \big[\mathcal{L}^{\alpha, \beta_i}[\tilde{v}(t,y)] - \phi(t, \alpha, B^T y + \sum_{i=1}^d D_i^T \beta_i) \big] = 0
\end{equation}
with terminal condition
\begin{equation}
    \tilde{v}(T,y) = - h(y) = - \frac12 (y^T + L^T) G^{-1} (y+ L).
\end{equation}
The generator is given by
\begin{equation}
    \mathcal{L}^{\alpha, \beta_i}[\tilde{v}(t, y)] = \bigg(\alpha^T - y^T A - \sum_{i=1}^d \beta_i^T C_i\bigg)D_y[\tilde{v}] + \frac12 \sum_{i=1}^d \beta_i^T D_y^2[\tilde{v}] \beta_i
\end{equation}
To find the supremum, we set the derivatives with respect to $\alpha, \beta_1, \dots, \beta_d$ to zero. We have
\begin{align}
    &D_\alpha \big[\mathcal{L}^{\alpha, \beta_i}[\tilde{v}(t,y)] - \phi\big] = D_y[\tilde{v}] - \tilde{Q}\alpha - \tilde{S}^T (B^T y + \sum_{i=1}^d D_i^T \beta_i) = 0 \label{eq: dual_sys1}\\
    &D_{\beta_i}\big[\mathcal{L}^{\alpha, \beta_i}[\tilde{v}(t,y)] - \phi\big] = - C_i D_y[\tilde{v}] + D_y^2[\tilde{v}] \beta_i
    - D_i (\tilde{S}\alpha + \tilde{R}(B^T y + \sum_{i=1}^d D_i^T \beta_i)) = 0 \label{eq: dual_sys2}
\end{align}
Solving this system, we can get the optimal controls, which we denote by $\alpha^\ast, \beta_i^\ast$.
%From the first equation, we get 
%\begin{equation*}
%    \beta = \tilde{S}^\dagger (D_y[\tilde{v}] - \tilde{Q} \alpha),
%\end{equation*}
%where $\tilde{S}^\dagger = (\tilde{S}\tilde{S}^T)^{-1} \tilde{S}$ is the Moore-Penrose inverse of $\tilde{S}^T$. From the second equation, we get
%\begin{align*}
%    \beta_i = (D_y^2[\tilde{v}])^{-1} \bigg[ (C_i + D_i \tilde{R}\tilde{S}^\dagger) D_y[\tilde{v}] + (D_i \tilde{S} - D_i \tilde{R}\tilde{S}^\dagger \tilde{Q})\alpha \bigg]
%\end{align*}
%Substituting into the first equation
%\begin{align*}
%    \tilde{Q} \alpha 
%    &= D_y[\tilde{v}] - \tilde{S}^T \beta\\
%    &= D_y[\tilde{v}] - \tilde{S}^T B^T Y - \tilde{S}^T \sum_{i=1}^d D_i^T \beta_i\\
%    &= D_y[\tilde{v}] - \tilde{S}^T B^T Y - \tilde{S}^T \sum_{i=1}^d D_i^T (D_y^2[\tilde{v}])^{-1} \bigg[ (C_i + D_i \tilde{R}\tilde{S}^\dagger) D_y[\tilde{v}] + (D_i \tilde{S} - D_i \tilde{R}\tilde{S}^\dagger \tilde{Q})\alpha \bigg]
%\end{align*}
%Rearranging we get
%\begin{equation}
%    \bigg(\tilde{Q} + \tilde{S}^T \sum_{i=1}^d D_i^T (D_y^2[\tilde{v}])^{-1} (D_i \tilde{S} - D_i\tilde{R}\tilde{S}^\dagger \tilde{Q}) \bigg)\alpha = D_y[\tilde{v}] - \tilde{S}^T B^T Y -\tilde{S}^T \sum_{i=1}^d D_i^T (D_y^2[\tilde{v}])^{-1} \big(D_i \tilde{R}\tilde{S}^\dagger + C_i\big) D_y[\tilde{v}] 
%\end{equation}
%so
%\begin{equation}
%    \alpha^\ast = \bigg[\tilde{Q} + \tilde{S}^T \sum_{i=1}^d D_i^T (D_y^2[\tilde{v}])^{-1} (D_i \tilde{S} - D_i\tilde{R}\tilde{S}^\dagger \tilde{Q}) \bigg]^{-1} \bigg[D_y[\tilde{v}] - \tilde{S}^T B^T Y -\tilde{S}^T \sum_{i=1}^d D_i^T (D_y^2[\tilde{v}])^{-1} \big(D_i \tilde{R}\tilde{S}^\dagger + C_i\big) D_y[\tilde{v}]  \bigg] \label{eq: dual_alpha1}
%\end{equation}
%and the optimal $\beta_i$ are
%\begin{equation}
%    \beta_i^\ast = (D_y^2[\tilde{v}])^{-1} \bigg[ (C_i + D_i \tilde{R}\tilde{S}^\dagger) D_y[\tilde{v}] + (D_i \tilde{S} - D_i \tilde{R}\tilde{S}^\dagger \tilde{Q})\alpha^\ast \bigg] \label{eq: dual_beta1}
%\end{equation}
The HJB equation then becomes
\begin{equation}
    \frac{\partial \tilde{v}}{\partial t} + \bigg({\alpha^\ast}^T - Y^T A - \sum_{i=1}^d {\beta_i^\ast}^T C_i\bigg)D_y[\tilde{v}] + \frac12 \sum_{i=1}^d {\beta_i^\ast}^T D_y^2[\tilde{v}] {\beta_i^\ast} - \phi\bigg(t, \alpha^\ast, B^T y + \sum_{i=1}^d D_i^T \beta_i^\ast \bigg) \label{eq: dual_hjb}
\end{equation}

%%%%%%%%%%%%%%%%%%%%%%%%%%%%%%%%%%%%%%%%%%%%%%%%%%%%
%%%%%%%%%%%%% SOLVING DUAL HJB%%%%%%%%%%%%%%%%%%%%%%
%%%%%%%%%%%%%%%%%%%%%%%%%%%%%%%%%%%%%%%%%%%%%%%%%%%%

\subsubsection{Solving the Dual HJB}
Suppose that $\tilde{v}$ is a quadratic function in $y$ and use the ansatz
\begin{equation}
    \tilde{v}(t,y) = \frac12 y^T \tilde{P}(t) y + y^T \tilde{M}(t) + \tilde{N}(t),
\end{equation}
with terminal conditions
\begin{equation}
    \tilde{P}(T) = -G^{-1}(T), \quad \tilde{M}(T) = - G^{-1}(T)L(T), \quad \tilde{N}(T) = \frac12 L^T(T)G^{-1}(T)L(T). \label{eq: dual_terminal_conditions}
\end{equation}
Then
\begin{align*}
    &\frac{\partial \tilde{v}}{\partial t}(t, y) = \frac12 y^T \frac{\d \tilde{P}}{\d t}(t) y + y^T \dot{\tilde{M}}(t) + \dot{\tilde{N}}(t)\\
    &D_y[\tilde{v(t,y)}] = \tilde{P}(t) y + \tilde{M}(t)\\
    &D_y^2[\tilde{v(t,y)}] = \tilde{P}(t)
\end{align*}
%Substituting this into the controls \eqref{eq: dual_alpha1} and \eqref{eq: dual_beta1}, we get
%\begin{align}
%    &\alpha^\ast = \bigg[\tilde{Q} + \tilde{S}^T \sum_{i=1}^d D_i^T \tilde{P}^{-1} (D_i \tilde{S} - D_i\tilde{R}\tilde{S}^\dagger \tilde{Q}) \bigg]^{-1} \bigg[\tilde{P}y+b - \tilde{S}^T B^T Y -\tilde{S}^T \sum_{i=1}^d D_i^T \tilde{P}^{-1} \big(D_i \tilde{R}\tilde{S}^\dagger + C_i\big) (\tilde{P}y+b)\bigg]\\
%    &\beta_i^\ast = \tilde{P}^{-1} \bigg[ (C_i + D_i \tilde{R}\tilde{S}^\dagger) (\tilde{P}y+b)+ (D_i \tilde{S} - D_i \tilde{R}\tilde{S}^\dagger \tilde{Q})\alpha^\ast \bigg]
%\end{align}
The system of equations \eqref{eq: dual_sys1} and \eqref{eq: dual_sys2} from which we derive the optimal controls $\alpha^\ast$ and $\beta_i^\ast$ is now given by 
\begin{align*}
    &\tilde{P}y + \tilde{M} - \tilde{Q}\alpha - \tilde{S}^T (B^T Y + \sum_{i=1}^d D_i^T \beta_i) = 0\\
    &C_i (\tilde{P}y + \tilde{M}) - \tilde{P} \beta_i
    + D_i (\tilde{S}\alpha + \tilde{R}(B^T Y + \sum_{i=1}^d D_i^T \beta_i)) = 0
\end{align*}
Clearly, the solutions for $\alpha^\ast$ and $\beta^\ast$ are linear in $y$, hence we denote by $\tilde{\theta}$ and $\tilde{\kappa}$ the coefficients before $y$ and the free coefficient in $\alpha^\ast$ and similarly for $\beta_i$, i.e. 
\begin{equation}
    \alpha^\ast = \tilde{\theta} y + \tilde{\kappa}, \quad \beta_i = \tilde{\theta}_i y + \tilde{\kappa}_i. \label{eq: dual_coeffs}
\end{equation}
Substituting this into the HJB equation \eqref{eq: dual_hjb} we get 
\begin{align}
     \frac12 y^T \frac{\d \tilde{P}}{\d t} y + y^T \dot{\tilde{M}} + \dot{\tilde{N}} + \bigg(y^T \tilde{\theta}^T + \tilde{\kappa}^T - y^T A - \sum_{i=1}^d (y^T \tilde{\theta}_i^T + \tilde{\kappa}_i^T) C_i\bigg)(\tilde{P}y+\tilde{M})\\
     + \frac12 \sum_{i=1}^d (y^T\tilde{\theta}_i^T + \tilde{\kappa}_i^T) \tilde{P} (\tilde{\theta}_i y + \tilde{\kappa}_i) - \phi\bigg(t, \tilde{\theta} y + \tilde{\kappa}, B^T y + \sum_{i=1}^d D_i^T (\tilde{\theta}_i y + \tilde{\kappa}_i) \bigg) = 0
\end{align}
Expanding $\phi\bigg(t, \tilde{\theta} y + \tilde{\kappa}, B^T y + \sum_{i=1}^d D_i^T (\tilde{\theta}_i y + \tilde{\kappa}_i) \bigg) $ we get
\begin{align*}
    \phi\bigg(t, \tilde{\theta} y + \tilde{\kappa}, B^T y + \sum_{i=1}^d D_i^T (\tilde{\theta}_i y + \tilde{\kappa}_i) \bigg) 
    = \frac12 (\tilde{\theta} y + \tilde{\kappa})^T \tilde{Q}(\tilde{\theta} y + \tilde{\kappa})+ (\tilde{\theta} y + \tilde{\kappa})^T \tilde{S}^T \big(B^T y + \sum_{i=1}^d D_i^T (\tilde{\theta}_i y + \tilde{\kappa}_i)\big)\\
    + \frac12\big(B^T y + \sum_{i=1}^d D_i^T (\tilde{\theta}_i y + \tilde{\kappa}_i)\big)^T \tilde{R}(B^T y + \sum_{i=1}^d D_i^T (\tilde{\theta}_i y + \tilde{\kappa}_i))
\end{align*}
Rearranging, we get 
\begin{align*}
     \frac12 y^T \frac{\d \tilde{P}}{\d t} y + y^T \frac{\d \tilde{M}}{\d t} + \frac{\d \tilde{N}}{\d t} + y^T\tilde{\theta}^T \tilde{P} y + y^T \tilde{P} \tilde{\kappa} - y^T A \tilde{P} y - y^T \sum_{i=1}^d \tilde{\theta}_i^T C_i \tilde{P} y - y^T \tilde{P} \sum_{i=1}^d C_i^T \tilde{\kappa}_i\\
     + y^T \tilde{\theta}^T \tilde{M}+ \tilde{\kappa}^T \tilde{M}- y^T A \tilde{M}- \sum_{i=1}^d y^T \tilde{\theta}_i^T C_i \tilde{M}- \sum_{i=1}^d \tilde{\kappa}_i^T C_i \tilde{M}+ \frac12 y^T \sum_{i=1}^d\tilde{\theta}_i^T \tilde{P} \tilde{\theta}_i y + y^T \sum_{i=1}^d\tilde{\theta}_i \tilde{P} \tilde{\kappa}_i + \frac12 \sum_{i=1}^d\tilde{\kappa}_i^T \tilde{P} \tilde{\kappa}_i\\
     - \frac12 y^T \tilde{\theta}^T \tilde{Q}\tilde{\theta} y - y^T \tilde{\theta}^T \tilde{Q}\tilde{\kappa} - \frac12 \tilde{\kappa}^T \tilde{Q}\tilde{\kappa} - y^T \tilde{\theta}^T \tilde{S}^T (B^T + \sum_{i=1}^d D_i^T \tilde{\theta}_i)y - y^T \tilde{\theta}^T \tilde{S}^T\sum_{i=1}^d D_i^T \tilde{\kappa}_i\\
     - y^T(B + \sum_{i=1}^d \tilde{\theta}_i^T D_i)\tilde{S}\tilde{\kappa} - \tilde{\kappa}^T \tilde{S}^T \sum_{i=1}^d D_i^T \tilde{\kappa}_i -\frac12 y^T\bigg(B + \sum_{i=1}^d\tilde{\theta}_i^T D_i\bigg) \tilde{R} \bigg(B^T + \sum_{i=1}^d D_i^T \tilde{\theta}_i\bigg)y\\
     - y^T\bigg(B + \sum_{i=1}^d\tilde{\theta}_i^T D_i\bigg) \tilde{R} \sum_{i=1}^d D_i^T \tilde{\kappa}_i - \frac12 \bigg(\sum_{i=1}^d\tilde{\kappa}_i^T D_i\bigg) \tilde{R} \bigg(\sum_{i=1}^d D_i^T \tilde{\kappa}_i\bigg) = 0
\end{align*}
Grouping together the coefficients in front of $y$ we get:
\begin{align*}
    &y^T \bigg[ \frac12 \frac{\d \tilde{P}}{\d t} + \tilde{\theta}^T \tilde{P} - A \tilde{P}  - \sum_{i=1}^d \tilde{\theta}_i^T C_i \tilde{P} + \frac12 \sum_{i=1}^d \tilde{\theta}_i^T \tilde{P} \tilde{\theta}_i - \frac12 \tilde{\theta}^T \tilde{Q}\tilde{\theta} - \tilde{\theta}^T \tilde{S}^T \big(B^T + \sum_{i=1}^d D_i^T \tilde{\theta}_i \big)\\
    &- \frac12 \bigg(B + \sum_{i=1}^d\tilde{\theta}_i^T D_i\bigg) \tilde{R} \bigg(B^T + \sum_{i=1}^d D_i^T \tilde{\theta}_i\bigg) \bigg]y + y^T \bigg[ \frac{\d \tilde{M}}{\d t} + \tilde{P} \tilde{\kappa} - \tilde{P} \sum_{i=1}^d C_i^T \tilde{\kappa}_i + \tilde{\theta}^T \tilde{M} - A\tilde{M} - \sum_{i=1}^d \tilde{\theta}_i^T C_i \tilde{M}\\
    &+ \sum_{i=1}^d\tilde{\theta}_i \tilde{P} \tilde{\kappa}_i - \tilde{\theta}^T \tilde{Q} \tilde{\kappa} - \tilde{\theta}^T \tilde{S}^T \sum_{i=1}^d D_i^T \tilde{\kappa}_i - \big(B + \sum_{i=1}^d\tilde{\theta}_i^T D_i \big)\tilde{S}\tilde{\kappa} - \big( B + \sum_{i=1}^d \tilde{\theta}_i^T D_i \big) \tilde{R} \sum_{i=1}^dD_i^T \tilde{\kappa}_i  \bigg]\\
    &+ \frac{\d \tilde{N}}{\d t} + \tilde{\kappa}^T \tilde{M} - \sum_{i=1}^d\tilde{\kappa}_i^T C_i \tilde{M} + \frac12 \sum_{i=1}^d\tilde{\kappa}_i^T \tilde{P} \tilde{\kappa}_i - \frac12 \tilde{\kappa}^T \tilde{Q}\tilde{\kappa} - \tilde{\kappa}^T \tilde{S}^T \sum_{i=1}^d D_i^T\tilde{\kappa}_i - \frac12 \bigg(\sum_{i=1}^d\tilde{\kappa}_i^T D_i\bigg) \tilde{R} \bigg(\sum_{i=1}^d D_i^T \tilde{\kappa}_i\bigg) = 0
\end{align*}
This equation must equal zero for all $y$, hence the coefficients in front of the quadratic term, as well as $y^T$ and the free coefficient must be zero. Setting the coefficients to zero, we get the system
\begin{align*}
    \frac12 \frac{\d \tilde{P}}{\d t} + \tilde{\theta}^T \tilde{P} - A \tilde{P}  - \sum_{i=1}^d \tilde{\theta}_i^T C_i \tilde{P} + \frac12 \sum_{i=1}^d \tilde{\theta}_i^T \tilde{P} \tilde{\theta}_i - \frac12 \tilde{\theta}^T \tilde{Q}\tilde{\theta} - \tilde{\theta}^T \tilde{S}^T \big(B^T + \sum_{i=1}^d D_i^T \tilde{\theta}_i \big)\\
    - \frac12 \bigg(B + \sum_{i=1}^d\tilde{\theta}_i^T D_i\bigg) \tilde{R} \bigg(B^T + \sum_{i=1}^d D_i^T \tilde{\theta}_i\bigg) = 0 \numberthis \label{eq: dual_hjb_sol1}\\
    \frac{\d \tilde{M}}{\d t} + \tilde{P} \tilde{\kappa} - \tilde{P} \sum_{i=1}^d C_i^T \tilde{\kappa}_i + \tilde{\theta}^T \tilde{M} - A\tilde{M} - \sum_{i=1}^d \tilde{\theta}_i^T C_i \tilde{M} 
    + \sum_{i=1}^d\tilde{\theta}_i a \tilde{\kappa}_i - \tilde{\theta}^T \tilde{Q} \tilde{\kappa} - \tilde{\theta}^T \tilde{S}^T \sum_{i=1}^d D_i^T \tilde{\kappa}_i\\
    - \big(B + \sum_{i=1}^d\tilde{\theta}_i^T D_i \big)\tilde{S}\tilde{\kappa} - \big( B + \sum_{i=1}^d \tilde{\theta}_i^T D_i \big) \tilde{R} \sum_{i=1}^dD_i^T \tilde{\kappa}_i = 0 \numberthis \label{eq: dual_hjb_sol2}\\
    \frac{\d \tilde{N}}{\d t} + \tilde{\kappa}^T \tilde{M} - \sum_{i=1}^d\tilde{\kappa}_i^T C_i \tilde{M} + \frac12 \sum_{i=1}^d\tilde{\kappa}_i^T \tilde{P} \tilde{\kappa}_i - \frac12 \tilde{\kappa}^T \tilde{Q}\tilde{\kappa} - \tilde{\kappa}^T \tilde{S}^T \sum_{i=1}^d D_i^T\tilde{\kappa}_i\\
    - \frac12 \bigg(\sum_{i=1}^d\tilde{\kappa}_i^T D_i\bigg) \tilde{R} \bigg(\sum_{i=1}^d D_i^T \tilde{\kappa}_i\bigg) = 0
\end{align*}
where $\tilde{\theta}, \tilde{\kappa}, \tilde{\theta}_i$ and $\tilde{\kappa}_i$ satisfy
\begin{align*}
    &\tilde{P}y + \tilde{M} - \tilde{Q}(\tilde{\theta}y + \tilde{\kappa}) - \tilde{S}^T (B^T Y + \sum_{i=1}^d D_i^T (\tilde{\theta}_i y + \tilde{\kappa}_i)) = 0\\
    &C_i (\tilde{P}y + \tilde{M}) - \tilde{P} (\tilde{\theta}_i y + \tilde{\kappa}_i)
    + D_i \tilde{S}(\tilde{\theta}y + \tilde{\kappa}) + D_i\tilde{R}\bigg(B^T Y + \sum_{i=1}^d D_i^T (\tilde{\theta}_i y + \tilde{\kappa}_i)\bigg) = 0
\end{align*}
with terminal conditions
\begin{equation}
    \tilde{P}(T) = -G^{-1}(T), \quad \tilde{M}(T) = - G^{-1}(T)L(T), \quad \tilde{N}(T) = \frac12 L^T(T)G^{-1}(T)L(T). 
\end{equation}

%%%%%%%%%%%%%%%%%%%%%%%%%%%%%%%%%%%%%%%%%%%%%%%%%%%%%%%%%%%%%%%%%%%%%
%%%%%%%%%%%%%%%%%%%% DUAL BSDE %%%%%%%%%%%%%%%%%%%%%%%%%%%%%%%%%%%%55
%%%%%%%%%%%%%%%%%%%%%%%%%%%%%%%%%%%%%%%%%%%%%%%%%%%%%%%%%%%%%%%%%%%%


\subsection{Dual BSDE}
The Hamiltonian $\tilde{\mathcal{H}}: [t_0, T] \times \R^n \times \R^{nd} \times \R^n \times \R^{nd} \to \R$ for the dual problem is defined as
\begin{align*}
    \tilde{\mathcal{H}}(t, Y, \alpha, \beta_1, \dots, \beta_d, p_2, q_2) 
    &= -\phi \bigg(t, \alpha, B^T Y + \sum_{i=1}^d D_i^T \beta_i \bigg) + p_2^T(\alpha - A^T Y - \sum_{i=1}^d C_i^T \beta_i) + \sum_{i=1}^d \beta_i^T q_{2, i} \\
    &= p_2^T\alpha - p_2^T A^T Y - p_2^T\sum_{i=1}^d C_i^T\beta_i   + \sum_{i=1}^d \beta_i^T q_{2,i} - \phi \bigg(t, \alpha, B^T Y + \sum_{i=1}^d D_i^T \beta_i \bigg) \numberthis \label{eq: dual_hamiltonian}
\end{align*}
The dual process together with the associated adjoin equation are given by the system
\begin{equation}
    \begin{cases}
        \d Y(t) &= \big[ \alpha - A^T Y - \sum_{i=1}^d C_i^T \beta_i \big]\d t + \sum_{i=1}^d \beta_i \d W_i\\
        Y(t_0) &= y\\
        \d p_2 &= -D_y [\tilde{\mathcal{H}}] \d t + \sum_{i=1}^d q_{2, i} \d W_i  \\
        p_2(T) &= - D_y[h(Y(T))]= - G^{-1} Y(T) - G^{-1} L
    \end{cases} \label{eq: fbsde_dual}
\end{equation}
The dual optimal control $(\hat{\alpha}, \hat{\beta_1}, \dots, \hat{\beta_d})$ of \eqref{eq: fbsde_dual} satisfies 
\begin{equation}
    \tilde{\mathcal{H}}(t, \hat{Y}(t), \hat{\alpha}(t), \hat{\beta_1}(t), \dots, \hat{\beta_d}(t), \hat{p}_2(t), \hat{q}_2(t) ) = \max_{\alpha, \beta_1 , \dots, \beta_d}\tilde{\mathcal{H}} (t, \hat{Y}(t), \hat{\alpha}, \hat{\beta_1}, \dots, \hat{\beta_d},\hat{p}_2(t), \hat{q}_2(t) )
\end{equation}
Since we have no constraints on the control, we find the maximum by setting $D_\alpha [\tilde{\mathcal{H}}] = 0$ and $D_{\beta_i} [\tilde{\mathcal{H}}] = 0$ for all $i = 1,\dots, d$, so
\begin{align}
    &D_\alpha[\tilde{\mathcal{H}}] = p_2 - \tilde{Q}\alpha - \tilde{S}^T \bigg(B^T Y + \sum_{i=1}D_i^T \beta_i \bigg) = 0  \label{eq: dual_hamiltonian_condition1}\\
    &D_{\beta_i}[\tilde{\mathcal{H}}] = q_{2,i} - C_i p_2 - D_i \tilde{S}\alpha - D_i \tilde{R}\bigg(B^T Y + \sum_{i=1}D_i^T \beta_i \bigg) = 0 \label{eq: dual_hamiltonian_condition2}
\end{align}
which we call the dual Hamiltonian condition. We try an ansatz for the control of the form
\begin{align*}
    \alpha = \tilde{\theta} Y + \tilde{\kappa}, \quad \beta_i = \tilde{\theta}_i Y + \tilde{\kappa}_i, \quad i \in \{1,\dots,d\}
\end{align*}
Substituting into the Hamiltonian \eqref{eq: dual_hamiltonian} we get
\begin{align*}
    \tilde{\mathcal{H}} = p_2^T(\tilde{\theta} Y + \tilde{\kappa}) - p_2^T A^T Y - p_2^T\sum_{i=1}^d C_i^T(\tilde{\theta}_i Y + \tilde{\kappa}_i)   + \sum_{i=1}^d (Y^T \tilde{\theta}_i^T + \tilde{\kappa}_i^T) q_{2,i}\\
    - \phi \bigg(t, \tilde{\theta} Y + \tilde{\kappa}, B^T Y + \sum_{i=1}^d D_i^T(\tilde{\theta}_i Y + \tilde{\kappa}_i)\bigg)
\end{align*}
The derivative of the dual Hamiltonian is then
\begin{align*}
    D_y[\tilde{\mathcal{H}}] = \tilde{\theta}^T p_2 - A p_2 - \sum_{i=1}^d \tilde{\theta}_i^T C_i p_2  + \sum_{i=1}^d \tilde{\theta}_i^T q_{2,i} - \tilde{\theta}^T \tilde{Q}\tilde{\theta} Y- \tilde{\theta}^T \tilde{Q} \tilde{\kappa} - 2 \tilde{\theta}^T \tilde{S}\bigg(B^T + \sum_{i=1}^d D_i^T \tilde{\theta}_i\bigg) Y\\
    - \tilde{\theta}^T \tilde{S}^T \sum_{i=1}^d D_i^T \tilde{\kappa}_i - \bigg( B + \sum_{i=1}^d \tilde{\theta}_i^T D_i \bigg) \tilde{S} \tilde{\kappa} - \bigg(B + \sum_{i=1}^d \tilde{\theta}_i^T D_i \bigg) \tilde{R} \bigg(B^T + \sum_{i=1}^d D_i^T \tilde{\theta}_i\bigg) Y - \bigg(B^T + \sum_{i=1}^d D_i^T \tilde{\theta}_i\bigg) \tilde{R}\sum_{i=1}^d D_i^T \tilde{\kappa}_i \numberthis \label{eq: dual_derivative_hamiltonian}
\end{align*}
Suppose also that
\begin{equation*}
    p_2 = \varphi(t) Y + \psi(t).
\end{equation*}
Applying Ito's formula to $p_2$, 
\begin{align*}
    \d p_2 &= (\dot{\varphi} Y + \dot{\psi})\d t + \varphi \d Y\\
    &= (\dot{\varphi} Y + \dot{\psi})\d t + \varphi \big[ \alpha - A^T Y - \sum_{i=1}^d C_i^T \beta_i \big]\d t + \varphi \sum_{i=1}^d \beta_i \d W_i\\
    &= \bigg[ \dot{\varphi}Y + \dot{\psi} + \varphi \alpha - \varphi A^T Y -\varphi \sum_{i=1}^d C_i^T \beta_i \bigg] \d t + \varphi \sum_{i=1}^d \beta_i \d W_i\\
    &= \bigg[ \dot{\varphi}Y + \dot{\psi} + \varphi \tilde{\theta} Y + \varphi \tilde{\kappa} - \varphi A^T Y - \varphi \sum_{i=1}^d C_i^T \tilde{\theta}_i Y - \varphi \sum_{i=1}^d C_i^T \tilde{\kappa}_i \bigg] \d t + \varphi \sum_{i=1}^d (\tilde{\theta}_i Y + \tilde{\kappa}_i) \d W_i \numberthis 
    \label{eq: dual_ito_p2}
\end{align*}
Equating the coefficients of \eqref{eq: dual_ito_p2} and \eqref{eq: fbsde_dual} we get 
\begin{align}
    &\dot{\varphi}Y + \dot{\psi} + \varphi \tilde{\theta} Y + \varphi \tilde{\kappa} - \varphi A^T Y - \varphi \sum_{i=1}^d C_i^T \tilde{\theta}_i Y - \varphi \sum_{i=1}^d C_i^T \tilde{\kappa}_i = -D_y[\tilde{\mathcal{H}}] \label{eq: dual_equal_coeff}\\
    &\varphi \tilde{\theta}_i Y + \varphi \tilde{\kappa}_i = q_{2, i} \label{eq: dual_equal_coeff2}\\
    &p_2 - \tilde{Q}(\tilde{\theta} Y + \tilde{\kappa}) - \tilde{S}^T \bigg(B^T Y + \sum_{i=1}^d D_i^T (\tilde{\theta}_i Y + \tilde{\kappa}_i)\bigg)= 0\\
    &q_{2,i} - C_i p_2 - D_i \tilde{S}(\tilde{\theta} Y + \tilde{\kappa}) - D_i \tilde{R}\bigg(B^T Y + \sum_{i=1}^d D_i^T (\tilde{\theta}_iY + \tilde{\kappa}_i)\bigg) = 0
\end{align}
where the last two equations are the Hamiltonian condition \eqref{eq: dual_hamiltonian_condition1} and \eqref{eq: dual_hamiltonian_condition2}, and the RHS of \eqref{eq: dual_equal_coeff} is given by \eqref{eq: dual_derivative_hamiltonian}. We now substitute $q_{2,i}$ from equation \eqref{eq: dual_equal_coeff2} into the rest and we get
\begin{align*}
    &\dot{\varphi}Y + \dot{\psi} + \varphi \tilde{\theta} Y + \varphi \tilde{\kappa} - \varphi A^T Y - \varphi \sum_{i=1}^d C_i^T \tilde{\theta}_i Y - \varphi \sum_{i=1}^d C_i^T \tilde{\kappa}_i = -\tilde{\theta}^T p_2 + A p_2 + \sum_{i=1}^d \tilde{\theta}_i^T C_i p_2\\
    &+ \sum_{i=1}^d \tilde{\theta}_i^T (\varphi \tilde{\theta}_i Y + \varphi \tilde{\kappa}_i) + \tilde{\theta}^T \tilde{Q}\tilde{\theta} Y + \tilde{\theta}^T \tilde{Q} \tilde{\kappa} + 2 \tilde{\theta}^T \tilde{S}\bigg(B^T + \sum_{i=1}^d D_i^T \tilde{\theta}_i\bigg) Y + \tilde{\theta}^T \tilde{S}^T \sum_{i=1}^d D_i^T \tilde{\kappa}_i\\
    &+ \bigg( B + \sum_{i=1}^d \tilde{\theta}_i^T D_i \bigg) \tilde{S} \tilde{\kappa}    + \bigg(B + \sum_{i=1}^d \tilde{\theta}_i^T D_i \bigg) \tilde{R} \bigg(B^T + \sum_{i=1}^d D_i^T \tilde{\theta}_i\bigg) Y + \bigg(B^T + \sum_{i=1}^d D_i^T \tilde{\theta}_i\bigg) \tilde{R}\sum_{i=1}^d D_i^T \tilde{\kappa}_i \numberthis \label{eq: dual1}\\
    &\varphi Y + \psi - \tilde{Q}(\tilde{\theta} Y + \tilde{\kappa}) - \tilde{S}^T \bigg(B^T Y + \sum_{i=1}^d D_i^T (\tilde{\theta}_i Y + \tilde{\kappa}_i)\bigg)= 0\numberthis \label{eq: dual2}\\
    &\varphi( \tilde{\theta}_i Y +  \tilde{\kappa}_i) - C_i (\varphi Y + \psi) - D_i \tilde{S}(\tilde{\theta} Y + \tilde{\kappa}) - D_i \tilde{R}\bigg(B^T Y + \sum_{i=1}^d D_i^T (\tilde{\theta}_iY + \tilde{\kappa}_i)\bigg) = 0 \numberthis \label{eq: dual3}
\end{align*}
From equations \eqref{eq: dual2} and \eqref{eq: dual3} one can compute the optimal controls $\alpha^\ast = \tilde{\theta}Y + \tilde{\kappa}$ and $\beta_i^\ast = \tilde{\theta}_i + \tilde{\kappa}_i$. 
%From \eqref{eq: dual2} we get 
%\begin{equation*}
%    \beta = \tilde{S}^\dagger (p_2 - \tilde{Q} \alpha),
%\end{equation*}
%so, plugging it into \eqref{eq: dual3}
%\begin{equation}
%    \beta_i = \varphi^{-1}(C_i p_2 + D_i \tilde{S}\alpha + D_i \tilde{R} \tilde{S}^\dagger p_2 - D_i \tilde{R}\tilde{S}^\dagger \tilde{Q} \alpha) \label{eq: dual4}
%\end{equation}
%Equation \eqref{eq: dual2} is now equivalent to
%\begin{align*}
%    p_2 &= \tilde{Q}\alpha + \tilde{S}^T B^T Y + \tilde{S}^T \sum_{i=1}^d D_i^T \beta_i \\
%    &=\tilde{Q}\alpha + \tilde{S}^T B^T Y + \tilde{S}^T \sum_{i=1}^d D_i^T \varphi^{-1}(C_i p_2 + D_i \tilde{S}\alpha + D_i \tilde{R} \tilde{S}^\dagger p_2 - D_i \tilde{R}\tilde{S}^\dagger \tilde{Q} \alpha)
%\end{align*}
%Rewriting this and plugging in $p_2 = \varphi Y + \psi$ we get 
%\begin{equation}
%    \alpha^\ast = \bigg[ \tilde{Q} +\tilde{S}^T \sum_{i=1}^d D_i^T \varphi^{-1}( D_i \tilde{S} - D_i \tilde{R}\tilde{S}^\dagger \tilde{Q}) \bigg]^{-1} \bigg[\varphi Y + \psi - \tilde{S}^TB^T Y - \tilde{S}^T \sum_{i=1}^d D_i^T \varphi^{-1}(C_i + D_i\tilde{R}\tilde{S}^\dagger)(\varphi Y + \psi)  \bigg]
%\end{equation}
%Now using \eqref{eq: dual4} we get 
%\begin{equation}
%    \beta_i^\ast = \varphi^{-1}\big[ C_i \varphi Y + C_i \psi + D_i \tilde{S}\alpha + D_i \tilde{R} \tilde{S}^\dagger (\varphi Y + \psi) - D_i \tilde{R}\tilde{S}^\dagger \tilde{Q} \alpha^\ast \big]
%\end{equation}
We rewrite equation \eqref{eq: dual1} as
\begin{align}
    \bigg[ \dot{\varphi} + \varphi \tilde{\theta} - \varphi A^T - \varphi \sum_{i=1}^d C_i^T \tilde{\theta}_i + \tilde{\theta}^T \varphi - A \varphi -\sum_{i=1}^d \tilde{\theta}_i C_i \varphi - \sum_{i=1}^d\tilde{\theta}_i^T\varphi \tilde{\theta}_i - \tilde{\theta}^T \tilde{Q}\tilde{\theta}\\
    - 2\tilde{\theta}^T \tilde{S}(B^T + \sum_{i=1}^d D_i^T \tilde{\theta}_i) - (B + \sum_{i=1}^d\tilde{\theta}_i^T D_i)\tilde{R} (B^T + \sum_{i=1}^d D_i^T \tilde{\theta}_i)\bigg]Y\\
    + \bigg[ \dot{\psi} + \varphi \tilde{\kappa} - \varphi \sum_{i=1}^d C_i^T \tilde{\kappa}_i + \tilde{\theta}^T \psi - A \psi - \sum_{i=1}^d \tilde{\theta}_i^T C_i \psi - \sum_{i=1}^d \tilde{\theta}_i^T \varphi \tilde{\kappa}_i \\
    -\tilde{\theta}^T \tilde{Q} \tilde{\kappa} - \tilde{\theta}^T \tilde{S}^T\sum_{i=1}^d D_i^T\tilde{\kappa}_i - (B + \sum_{i=1}^d\tilde{\theta}_i^T D_i) (\tilde{S}\tilde{\kappa} + \tilde{R} \sum_{i=1}^d D_i^T \tilde{\kappa}_i)\bigg] = 0
\end{align}
Since this must be true for all $Y$, the coefficient in front of $Y$ must be equal to zero, so we get
\begin{align*}
    \dot{\varphi} + \varphi \tilde{\theta} - \varphi A^T - \varphi \sum_{i=1}^d C_i^T \tilde{\theta}_i + \tilde{\theta}^T \varphi - A \varphi -\sum_{i=1}^d \tilde{\theta}_i C_i \varphi - \sum_{i=1}^d\tilde{\theta}_i^T\varphi \tilde{\theta}_i - \tilde{\theta}^T \tilde{Q}\tilde{\theta}\\
    - 2\tilde{\theta}^T \tilde{S}(B^T + \sum_{i=1}^d D_i^T \tilde{\theta}_i) - (B + \sum_{i=1}^d\tilde{\theta}_i^T D_i)\tilde{R} (B^T + \sum_{i=1}^d D_i^T \tilde{\theta}_i) = 0 \numberthis 
    \label{eq: dual_bsde_sol1}\\
    \dot{\psi} + \varphi \tilde{\kappa} - \varphi \sum_{i=1}^d C_i^T \tilde{\kappa}_i + \tilde{\theta}^T \psi - A \psi - \sum_{i=1}^d \tilde{\theta}_i^T C_i \psi - \sum_{i=1}^d \tilde{\theta}_i^T \varphi \tilde{\kappa}_i\\
    -\tilde{\theta}^T \tilde{Q} \tilde{\kappa} - \tilde{\theta}^T \tilde{S}^T\sum_{i=1}^d D_i^T\tilde{\kappa}_i - (B + \sum_{i=1}^d\tilde{\theta}_i^T D_i) (\tilde{S}\tilde{\kappa} + \tilde{R} \sum_{i=1}^d D_i^T \tilde{\kappa}_i) = 0, \numberthis \label{eq: dual_bsde_sol2}
\end{align*}
where $\tilde{\theta}, \tilde{\kappa}, \tilde{\theta}_i,$ and $\tilde{\kappa}_i$ satisfy the Hamiltonian condition:
\begin{align*}
    &\varphi Y + \psi - \tilde{Q}(\tilde{\theta} Y + \tilde{\kappa}) - \tilde{S}^T \bigg(B^T Y + \sum_{i=1}^d D_i^T (\tilde{\theta}_i Y + \tilde{\kappa}_i)\bigg)= 0\\
    &\varphi( \tilde{\theta}_i Y +  \tilde{\kappa}_i) - C_i (\varphi Y + \psi) - D_i \tilde{S}(\tilde{\theta} Y + \tilde{\kappa}) - D_i \tilde{R}\bigg(B^T Y + \sum_{i=1}^d D_i^T (\tilde{\theta}_iY + \tilde{\kappa}_i)\bigg) = 0
\end{align*}
and the terminal conditions are given by
\begin{equation*}
    \varphi(T) = - G^{-1}(T), \quad \psi(T) = - G^{-1}(T) L(T).
\end{equation*}


%%%%%%%%%%%%%%%%%%%%%%%%%%%%%%%%%%%%%%%%%%%%%%%%%%%%%%%%%%%%%%%%%%%%%%%%%5
%%%%%%%%%%%%%%%%%%%%%%%%%%%%% EQUIVALENCE DUAL HJB AND DUAL BSDE
%%%%%%%%%%%%%%%%%%%%%%%%%%%%%%%%%%%%%%%%%%%%%%%%%%%%%%%%%%%%%%%%%%%%%%%%%%
\subsection{Equivalence between Dual HJB and Dual BSDE}
From the dual HJB we get that $\alpha^\ast = \tilde{\theta}y + \tilde{\kappa}$ and $\beta^\ast = \tilde{\theta}_i y + \tilde{\kappa}_i$, such that $\tilde{\theta}, \tilde{\kappa}, \tilde{\theta}_i$ and $\tilde{\kappa}_i$ satisfy 
\begin{align*}
    &\tilde{P}y + \tilde{M} - \tilde{Q}(\tilde{\theta}y + \tilde{\kappa}) - \tilde{S}^T (B^T Y + \sum_{i=1}^d D_i^T (\tilde{\theta}_i y + \tilde{\kappa}_i)) = 0\\
    &C_i (\tilde{P}y + \tilde{M}) - \tilde{P} (\tilde{\theta}_i y + \tilde{\kappa}_i)
    + D_i \tilde{S}(\tilde{\theta}y + \tilde{\kappa}) + D_i\tilde{R}\bigg(B^T Y + \sum_{i=1}^d D_i^T (\tilde{\theta}_i y + \tilde{\kappa}_i)\bigg) = 0
\end{align*}
Similarly, from the dual BSDE $\tilde{\theta}, \tilde{\kappa}, \tilde{\theta}_i$ and $\tilde{\kappa}_i$ satisfy 

\begin{align*}
    &\varphi Y + \psi - \tilde{Q}(\tilde{\theta} Y + \tilde{\kappa}) - \tilde{S}^T \bigg(B^T Y + \sum_{i=1}^d D_i^T (\tilde{\theta}_i Y + \tilde{\kappa}_i)\bigg)= 0\\
    &\varphi( \tilde{\theta}_i Y +  \tilde{\kappa}_i) - C_i (\varphi Y + \psi) - D_i \tilde{S}(\tilde{\theta} Y + \tilde{\kappa}) - D_i \tilde{R}\bigg(B^T Y + \sum_{i=1}^d D_i^T (\tilde{\theta}_iY + \tilde{\kappa}_i)\bigg) = 0
\end{align*}
The solutions of these equations are the same if and only  if we have the relation
\begin{equation*}
    \varphi = \tilde{P}, \quad \psi = \tilde{M}.
\end{equation*}
The first ODE from the dual BSDE is \eqref{eq: dual_bsde_sol1}, so substituting $\tilde{P} = \varphi$ in it we get 
\begin{align*}
    \frac{\d \tilde{P}}{\d t} + \tilde{P} \tilde{\theta} - \tilde{P} A^T - \tilde{P} \sum_{i=1}^d C_i^T \tilde{\theta}_i + \tilde{\theta}^T \tilde{P} - A \tilde{P} -\sum_{i=1}^d \tilde{\theta}_i C_i \tilde{P} - \sum_{i=1}^d\tilde{\theta}_i^T\tilde{P} \tilde{\theta}_i - \tilde{\theta}^T \tilde{Q}\tilde{\theta}\\
    - 2\tilde{\theta}^T \tilde{S}(B^T + \sum_{i=1}^d D_i^T \tilde{\theta}_i) - (B + \sum_{i=1}^d\tilde{\theta}_i^T D_i)\tilde{R} (B^T + \sum_{i=1}^d D_i^T \tilde{\theta}_i) = 0
\end{align*}
The first ODE from the dual HJB equation is given by \eqref{eq: dual_hjb_sol1}:
\begin{align*}
    \frac12 \frac{\d \tilde{P}}{\d t} + \tilde{\theta}^T \tilde{P} - A \tilde{P}  - \sum_{i=1}^d \tilde{\theta}_i^T C_i \tilde{P} + \frac12 \sum_{i=1}^d \tilde{\theta}_i^T \tilde{P} \tilde{\theta}_i - \frac12 \tilde{\theta}^T \tilde{Q}\tilde{\theta} - \tilde{\theta}^T \tilde{S}^T \big(B^T + \sum_{i=1}^d D_i^T \tilde{\theta}_i \big)\\
    - \frac12 \bigg(B + \sum_{i=1}^d\tilde{\theta}_i^T D_i\bigg) \tilde{R} \bigg(B^T + \sum_{i=1}^d D_i^T \tilde{\theta}_i\bigg) = 0
\end{align*}
The two equations are equivalent with the second being the first one divided by $2$. Similarly, plugging in $\tilde{P} = \varphi$ and $\tilde{M} = \psi$ in \eqref{eq: dual_bsde_sol2} we get
\begin{align*}
    \frac{\d \tilde{M}}{\d t} + \tilde{P} \tilde{\kappa} - \tilde{P} \sum_{i=1}^d C_i^T \tilde{\kappa}_i + \tilde{\theta}^T \tilde{M} - A \tilde{M} - \sum_{i=1}^d \tilde{\theta}_i^T C_i \tilde{M} - \sum_{i=1}^d \tilde{\theta}_i^T \tilde{P} \tilde{\kappa}_i-\tilde{\theta}^T \tilde{Q} \tilde{\kappa} - \tilde{\theta}^T \tilde{S}^T\sum_{i=1}^d D_i^T\tilde{\kappa}_i\\
    - (B + \sum_{i=1}^d\tilde{\theta}_i^T D_i) (\tilde{S}\tilde{\kappa} + \tilde{R} \sum_{i=1}^d D_i^T \tilde{\kappa}_i) = 0
\end{align*}
The respective ODE from the dual HJB is \eqref{eq: dual_hjb_sol2}:
\begin{align*}
    \frac{\d \tilde{M}}{\d t} + \tilde{P} \tilde{\kappa} - \tilde{P} \sum_{i=1}^d C_i^T \tilde{\kappa}_i + \tilde{\theta}^T \tilde{M} - A\tilde{M} - \sum_{i=1}^d \tilde{\theta}_i^T C_i \tilde{M} 
    + \sum_{i=1}^d\tilde{\theta}_i a \tilde{\kappa}_i - \tilde{\theta}^T \tilde{Q} \tilde{\kappa} - \tilde{\theta}^T \tilde{S}^T \sum_{i=1}^d D_i^T \tilde{\kappa}_i\\
    - \big(B + \sum_{i=1}^d\tilde{\theta}_i^T D_i \big)\tilde{S}\tilde{\kappa} - \big( B + \sum_{i=1}^d \tilde{\theta}_i^T D_i \big) \tilde{R} \sum_{i=1}^dD_i^T \tilde{\kappa}_i = 0
\end{align*}
As we can see the equations are identical, and their terminal conditions are also the same, so the two methods are equivalent. 

\subsection{Equivalence of Primal HJB and Dual HJB}
Recall that the optimal value of the dual problem was the solution to 
\begin{equation*}
    \inf_y (x^T y - \tilde{v}(y)),
\end{equation*}
Substituting the respective ansatz, we have
\begin{equation*}
    \frac12 x^T P x + x^T M + N = \inf_y \bigg\{ x^T y - \frac12 y^T \tilde{P} y - y^T \tilde{M} - \tilde{N}  \bigg\}
\end{equation*}
Setting the derivative of the RHS to zero, we get 
\begin{equation*}
    y = \tilde{P}^{-1}(x - \tilde{M}),
\end{equation*}
so
\begin{equation*}
    \frac12 x^T P x + x^T M + N  = - \frac12 (x^T - \tilde{M}^T)\tilde{P}^{-1} (x-\tilde{M}) - (x^T - \tilde{M}^T)\tilde{P}^{-1} \tilde{M} - \tilde{N} + x^T \tilde{P}^{-1}(x - \tilde{M})
\end{equation*}
Simplifying we get
\begin{equation*}
    \frac12 x^T P x + x^T M + N = \frac12 x^T \tilde{P}^{-1}x - x^T \tilde{P}^{-1}\tilde{M} + \frac12 \tilde{M}^T \tilde{P}^{-1} \tilde{M} - \tilde{N}.
\end{equation*}
Therefore, we get the relation
\begin{equation*}
    P = \tilde{P}^{-1}, \quad M = -\tilde{P}^{-1} \tilde{M}, \quad N =  \frac12 \tilde{M}^T \tilde{P}^{-1} \tilde{M} - \tilde{N}.
\end{equation*}
Consider a simpler case where $d=1$. % and $S = 0$. Then $\tilde{S}=0, \tilde{Q}= Q^{-1}, \tilde{R} = R^{-1}$.
Recall that the Riccati equation from the primal problem is given by \eqref{eq: primal_hjb_ricatti_1}
\begin{equation*}
     \dot{P}  +  2P A + 2P B \vartheta_1 + (C_1^T + \vartheta_1^T D_1^T)P(C_1 + D_1 \vartheta_1) 
     -  Q -  \vartheta_1^T R \vartheta_1 = 0,
\end{equation*}
where 
\begin{align*}
    \vartheta_1 = ( D_1^T  P D_1 - R)^{-1} ( - B^T  P - D_1^T  P C_1 )
\end{align*}
Therefore, our equation becomes
\begin{align*}
    0 = \frac{\d P}{\d t} + 2 PA - Q + C_1^T P C_1 + 2 (P B + C_1^T P D_1) \theta_1 + \theta_1^T (D_1^T P D_1 - R)\theta_1\\
    = \frac{\d P}{\d t} + 2 PA - Q + C_1^T P C_1 - (P B + C_1^T P D_1) ( D_1^T  P D_1 - R)^{-1} (B^T  P + D_1^T  P C_1 )
\end{align*}
Substituting $P = \tilde{P}^{-1}$, we get 
\begin{align*}
     \tilde{P}^{-1} \frac{\d \tilde{P}}{\d t} \tilde{P}^{-1} - 2 \tilde{P}^{-1} A + Q - C_1^T \tilde{P}^{-1} C_1 + (\tilde{P}^{-1} B + C_1^T \tilde{P}^{-1} D_1) ( D_1^T  \tilde{P}^{-1} D_1 - R)^{-1} (B^T  \tilde{P}^{-1} + D_1^T  \tilde{P}^{-1} C_1 ) = 0
\end{align*}
Multiplying on the left and on the right by $\tilde{P}$, we get 
\begin{align*}
    \frac{\d \tilde{P}}{\d t} - 2A\tilde{P} + \tilde{P}Q\tilde{P} - \tilde{P}C_1^T \tilde{P}^{-1} C_1\tilde{P} + \tilde{P}(\tilde{P}^{-1} B + C_1^T \tilde{P}^{-1} D_1) ( D_1^T  \tilde{P}^{-1} D_1 - R)^{-1} (B^T  \tilde{P}^{-1} + D_1^T  \tilde{P}^{-1} C_1 )\tilde{P} = 0 
\end{align*}
Rewriting this, we get 
\begin{align*}
    \frac{\d \tilde{P}}{\d t} - 2A\tilde{P} + \tilde{P}Q\tilde{P}  + \tilde{P}C_1^T (\Tilde{P}^{-1}D_1 (D_1^T \Tilde{P} D_1 - R)^{-1} D_1^T \Tilde{P}^{-1} - \tilde{P}^{-1} )C_1\tilde{P}+  B(D_1^T \Tilde{P} D_1 - R)B^T \\
    + 2B(D_1^T \Tilde{P} D_1 - R)^{-1} D_1^t \Tilde{P}^{-1} C_1 \Tilde{P}= 0 \numberthis \label{eq: equiv1}
\end{align*}
On the other hand, for the dual problem we have:
\begin{align*}
    \tilde{\theta} = Q \tilde{P}, \quad \tilde{\theta}_1 = (\tilde{P} - D_1 R^{-1} D_1^T)^{-1} (C_1 \tilde{P} + D_1 R^{-1}B^T)
\end{align*}
The dual Riccati equation is \eqref{eq: dual_hjb_sol1}:
\begin{align*}
   \frac{\d \tilde{P}}{\d t} + 2\tilde{\theta}^T \tilde{P} - 2A \tilde{P}  - 2\tilde{\theta}_1^T C_1 \tilde{P} + \tilde{\theta}_1^T \tilde{P} \tilde{\theta}_1 - \tilde{\theta}^T Q^{-1}\tilde{\theta} - (B + \tilde{\theta}_1^T D_1) R^{-1} (B^T + D_1^T \tilde{\theta}_1) = 0
\end{align*}
Substituting for $\tilde{\theta}$ we get
\begin{align*}
     \frac{\d \tilde{P}}{\d t} + \tilde{P}Q \tilde{P} -2 A \tilde{P}  - 2 \Tilde{P} C_1^T\tilde{\theta}_1 +  \tilde{\theta}_1^T \tilde{P} \tilde{\theta}_1
    - (B +\tilde{\theta}_1^T D_1) R^{-1} (B^T + D_1^T \tilde{\theta}_1) = 0
\end{align*}
Substituting for $\tilde{\theta}_1$ we get 
\begin{align*}
    0 = \frac{\d \tilde{P}}{\d t} + \tilde{P}Q \tilde{P} -2 A \tilde{P}  - 2 \Tilde{P} C_1^T\tilde{\theta}_1 +  \tilde{\theta}_1^T \tilde{P} \tilde{\theta}_1
    - (B +\tilde{\theta}_1^T D_1) R^{-1} (B^T + D_1^T \tilde{\theta}_1)\\
    =\frac{\d \tilde{P}}{\d t} + \tilde{P}Q \tilde{P} -2 A \tilde{P}  - 2 \Tilde{P} C_1^T\tilde{\theta}_1 +  \tilde{\theta}_1^T \tilde{P} \tilde{\theta}_1
    - B R^{-1}B^T - 2 B R^{-1} D_1^T \tilde{\theta}_1 - \tilde{\theta}_1^T D_1 R^{-1} D_1^T \tilde{\theta}_1\\
    =\frac{\d \tilde{P}}{\d t} + \tilde{P}Q \tilde{P} -2 A \tilde{P}  
    - B R^{-1}B^T - 2 (B R^{-1} D_1^T + \Tilde{P} C_1^T)\tilde{\theta}_1 + \tilde{\theta}_1^T (\Tilde{P} - D_1 R^{-1} D_1^T) \tilde{\theta}_1\\
    = \frac{\d \tilde{P}}{\d t} + \tilde{P}Q \tilde{P} -2 A \tilde{P}  
    - B R^{-1}B^T -  (B R^{-1} D_1^T + \Tilde{P} C_1^T)(\tilde{P} - D_1 R^{-1} D_1^T)^{-1} (C_1 \tilde{P} + D_1 R^{-1}B^T)
\end{align*}
This is then rewritten as 
\begin{align*}
    \frac{\d \Tilde{P}}{\d t} - 2 A \Tilde{P} + \Tilde{P} Q \Tilde{P} + \Tilde{P} C_1^T (D_1 R^{-1} D_1^T - \Tilde{P})^{-1} C_1 \Tilde{P} + 2 B R^{-1} D_1^T (D_1 R^{-1} D_1^T - \Tilde{P})^{-1} C_1 \Tilde{P}\\
    + B (R^{-1}D_1^T (D_1 R^{-1}D_1^T - \Tilde{P})^{-1} D_1 R^{-1} - R^{-1}) B^T 
     = 0 \numberthis \label{eq: equiv2}
\end{align*}
Now noting that
\begin{align*}
    (D_1 R^{-1} D_1^T - \Tilde{P})^{-1} = (\Tilde{P}^{-1}D_1 (D_1^T \Tilde{P} D_1 - R)^{-1} D_1^T \Tilde{P}^{-1} - \tilde{P}^{-1} )\\
    (R^{-1}D_1^T (D_1 R^{-1}D_1^T - \Tilde{P})^{-1} D_1 R^{-1} - R^{-1}) = (D_1^T \Tilde{P} D_1 - R)\\
    R^{-1} D_1^T (D_1 R^{-1} D_1^T - \Tilde{P})^{-1}= (D_1^T \Tilde{P} D_1 - R)^{-1} D_1^t \Tilde{P}^{-1}
\end{align*}
we get exactly \eqref{eq: equiv1}, so the dual and primal HJB methods are equivalent. 

